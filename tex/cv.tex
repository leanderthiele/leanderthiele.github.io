\documentclass[margin]{res}
\textwidth=5.2in
\voffset=-2cm
\usepackage[utf8]{inputenc}
\usepackage{eurosym}
\usepackage[usenames,dvipsnames]{xcolor}
\usepackage{hyperref}

%\pagestyle{headings}
%\markright{\hspace*{-3.3cm}\texttt{Leander F. Thiele, applying to Department of Physics}\hfill}

\hypersetup{colorlinks=false,allbordercolors=Apricot,pdfborderstyle={/S/U/W 1}}


\begin{document} 

\vspace*{0.8cm}

\name{Leander F. Thiele\\[12pt]}

\begin{resume} 


\vspace*{-0.5cm}
\section{Contact}
\href{mailto:leander.thiele@ipmu.jp}{\texttt{leander.thiele@ipmu.jp}},
\href{https://leanderthiele.github.io}{\texttt{leanderthiele.github.io}}

%\section{Research \\ Interests}
%	\begin{itemize} \itemsep -2pt
%		\item cosmology: theory, simulations, and data analysis
%		\item matter distribution as a probe of fundamental physics
%		\item optimal statistical use of cosmological tracers
%		\item astroparticle phenomenology
%	\end{itemize}

%\section{Summary of \\ Qualifications}
%	\begin{itemize} \itemsep -2pt
%		\item two major research projects in cosmology
%		%\item \textbf{add papers here}
%		\item strong theoretical background in cosmology \& particle physics
%		\item extensive experience in data analysis using Python and astronomy software
%		\item 1 publication in cosmology (1st author), 2 publications in biophysics
%	\end{itemize}

\section{Employment}
  {\bf University of Tokyo}, Project Assistant Professor \hfill 2024 -- present
  \begin{itemize}
    \item in the Center for Data-Driven Discovery at Kavli-IPMU
  \end{itemize}

\section{Education}
	{\bf Princeton University,} PhD Physics \hfill 2019 -- 2024
	\begin{itemize} \itemsep -2pt
		\item advisor: David N. Spergel
		\item graduate courses: QFT I, Extragalactic Astronomy,
		Cosmology, General Relativity, Condensed Matter
	\end{itemize}
	
	{\bf Perimeter Institute for Theoretical Physics,} MSc \hfill 2018 -- 2019
	\begin{itemize} \itemsep -2pt
		\item advisor: Kendrick M. Smith, co-advisor: J. Colin Hill
		\item thesis: \emph{Capturing non-Gaussianity: Analytic model for the one-point
		probability distribution function of cosmological fields within the halo model}
%		\begin{quote}\footnotesize
%		A powerful observable for cosmological fields primarily sourced by
%		massive halos is their one-point probability distribution function.
%		We present a general model for the one-point PDF, taking into account
%		previously neglected corrections. Application of the model
%		to the thermal Sunyaev-Zel’dovich PDF is successfully confirmed by
%		comparison to simulations. Preliminary results concerning the weak
%		lensing convergence seem to indicate that our model is in fact more
%		precise than the simulations.
%		\end{quote}
		\item graduate courses: QFT I \& II, Statistical Mechanics, Condensed Matter,
		Cosmology, General Relativity, Machine Learning
	\end{itemize}

	{\bf University of Oxford,} Physics, BA First Class \hfill 2015 -- 2018
	\begin{itemize} \itemsep -2pt
		\item ranked top of the cohort ($\sim 130$ students) in years 2 and 3
	\end{itemize}

\section{Publications}
{\bf L. Thiele}, E. Massara, A. Pisani, C. Hahn, D.N. Spergel, S. Ho, B. Wandelt,
\emph{Neutrino mass constraint from an Implicit Likelihood Analysis of BOSS voids}, 2023,\\
\href{https://arxiv.org/abs/2307.07555}{\mbox{\texttt{arXiv:2307.07555 [astro-ph.CO]}}}

{\bf L. Thiele}, G.A. Marques, J. Liu, M. Shirasaki,
\emph{Cosmological constraints from HSC Y1 lensing convergence PDF}, 2023,\\
\href{https://arxiv.org/abs/2304.05928}{\mbox{\texttt{arXiv:2304.05928 [astro-ph.CO]}}}

A.M. Delgado, D. Angl\'es-Alc\'azar, {\bf L. Thiele}, M. Ntampaka, S. Pandey,
K. Lehman, R.S. Somerville, S. Genel, F. Villaescusa-Navarro,
\emph{Predicting the impact of feedback on matter clustering with machine learning
      in CAMELS}, 2023,\\
\href{https://arxiv.org/abs/2301.02231}{\mbox{\texttt{arXiv:2301.02231 [astro-ph.GA]}}}

D. Wadekar, {\bf L. Thiele}, J.C. Hill, S. Pandey, F. Villaescusa-Navarro,
D.N. Spergel, M. Cranmer, D. Nagai, D. Angl\'es-Alc\'azar, S. Ho, L. Hernquist,
\emph{The SZ flux-mass (Y-M) relation at low halo masses: improvements with
      symbolic regression and strong constraints on baryonic feedback}, 2022,\\
MNRAS 522, 2,
\href{https://arxiv.org/abs/2209.02075}{\mbox{\texttt{arXiv:2209.02075 [astro-ph.CO]}}}

B.K.K. Lee, W. Coulton, {\bf L. Thiele}, S. Ho,
\emph{An exploration of the properties of cluster profiles for the thermal
      and kinetic Sunyaev-Zel'dovich effects}, 2022,\\
MNRAS 517, 420,
\href{https://arxiv.org/abs/2205.01710}{\mbox{\texttt{arXiv:2205.01710 [astro-ph.CO]}}}

{\bf L. Thiele}, M. Cranmer, W. Coulton, S. Ho, D.N. Spergel,
\emph{Predicting the Thermal Sunyaev-Zel'dovich Field
      using Modular and Equivariant Set-Based Neural Networks}, 2022,\\
MLST 3, 035002,
\href{https://arxiv.org/abs/2203.00026}{\mbox{\texttt{arXiv:2203.00026 [astro-ph.CO]}}},
poster presented at the Fourth Workshop on Machine Learning and the Physical Sciences (NeurIPS 2021)

{\bf L. Thiele}, D. Wadekar, J.C. Hill, N. Battaglia, J. Chluba,
F. Villaescusa-Navarro, L. Hernquist, M. Vogelsberger, D. Angl\'es-Alc\'azar,
F. Marinacci,
\emph{Percent-level constraints on baryonic feedback with spectral distortion
      measurements}, 2022,\\
Phys Rev D 105, 083505,
\href{https://arxiv.org/abs/2201.01663}{\mbox{\texttt{arXiv:2201.01663 [astro-ph.CO]}}}

D. Wakekar, {\bf L. Thiele}, F. Villaescusa-Navarro, J.C. Hill, D.N. Spergel,
M. Cranmer, N. Battaglia, D. Angl\'es-Alc\'azar, L. Hernquist, S. Ho,
\emph{Augmenting astrophysical scaling relations with machine learning:
      application to reducing the SZ flux-mass scatter}, 2022,\\
PNAS 120(12),
\href{https://arxiv.org/abs/2201.01305}{\mbox{\texttt{arXiv:2201.01305 [astro-ph.CO]}}}

\newpage
\vspace*{0.5cm}

F. Villaescusa-Navarro, S. Genel, D. Angl\'es-Alc\'azar,
L.A. Perez, P. Villanueva-Domingo, D. Wadekar, H. Shao,
F.G. Mohammad, S. Hassan, E. Moser, E.T. Lau, L.F.M.P. Valle, A. Nicola,
{\bf L. Thiele}, Y. Jo, O.H.E. Philcox, B.D. Oppenheimer, M. Tillman, C. Hahn,
N. Kaushal, A. Pisani, M. Gebhardt, A.M. Delgado, J. Caliendo,
C. Kreisch, K.W.K. Wong, W.R. Coulton, M. Eickenberg,
G. Parimbelli, Y. Ni, U.P. Steinwandel, V. La Torre,
R. Dave, N. Battaglia, D. Nagai, D.N. Spergel, L. Hernquist, B. Burkhart,
D. Narayanan, B. Wandelt, R.S. Somerville, G.L. Bryan, M. Viel, Y. Li, V. Irsic,
K. Kraljic, M. Vogelsberger,
\emph{The CAMELS project: public data release}, 2022,\\
\href{https://arxiv.org/abs/2201.01300}{\mbox{\texttt{arXiv:2201.01300 [astro-ph.CO]}}}

B. Maffei, M.H. Abitbol, N. Aghanim, J. Aumont, E. Battistelli, J. Chluba,
X. Coulon, P. De Bernardis, M. Douspis, J. Grain, S. Gervasoni, J.C. Hill,
A. Kogut, S. Masi, T. Matsumara, C. O Sullivan, L. Pagano, G. Pisano,
M. Remazeilles, A. Ritacco, A. Rotti, V. Sauvage, G. Savini, S.L. Stever,
A. Tartari, {\bf L. Thiele}, N. Trappe,
\emph{BISOU: a balloon project to measure the CMB spectral distortions}, 2021,\\
16th Marcel Grossmann Meeting,
\href{https://arxiv.org/abs/2111.00246}{\mbox{\texttt{arXiv:2111.00246 [astro-ph.IM]}}}

F. Villaescusa-Navarro, S. Genel, D. Angl\'es-Alc\'azar, {\bf L. Thiele},
R. Dave, D. Narayanan, A. Nicola, Y. Li, P. Villanueva-Domingo, B. Wandelt,
D.N. Spergel, R.S. Somerville, J.M. Zorrilla Matilla, F.G. Mohammad, S. Hassan,
H. Shao, D. Wadekar, M. Eickenberg, K.W.K. Wong, G. Contardo, Y. Jo, E. Moser,
E.T. Lau, L.F.M.P. Valle, L.A.~Perez, D. Nagai, N. Battaglia, M. Vogelsberger,
\emph{The CAMELS Multifield Dataset: Learning the Universe's Fundamental
      Parameters with Artificial Intelligence}, 2021,\\
Astrophys J Suppl Ser 259, 61,
\href{https://arxiv.org/abs/2109.10915}{\mbox{\texttt{arXiv:2109.10915 [cs.LG]}}}

F. Villaescusa-Navarro, S. Genel, D. Angl\'es-Alc\'azar, D.N. Spergel, Y. Li,
B. Wandelt, {\bf L. Thiele}, A. Nicola, J.M. Zorilla Matilla, H. Shao,
S. Hassan, D. Narayanan, R. Dave, M. Vogelsberger,
\emph{Robust marginalization of baryonic effects for cosmological inference
      at the field level}, 2021,\\
\href{https://arxiv.org/abs/2109.10360}{\mbox{\texttt{arXiv:2109.10360 [astro-ph.CO]}}}

F. Villaescusa-Navarro, D. Angl\'es-Alc\'azar, S. Genel, D.N. Spergel, Y. Li,
B. Wandelt, A.~Nicola, {\bf L. Thiele}, S. Hassan, J.M. Zorrilla Mattilla,
D. Narayanan, R. Dave, M.~Vogelsberger,
\emph{Multifield Cosmology with Artificial Intelligence}, 2021,\\
\href{https://arxiv.org/abs/2109.09747}{\mbox{\texttt{arXiv:2109.09747 [astro-ph.CO]}}}

{\bf L. Thiele}, Y. Guan, J.C. Hill, A. Kosowsky, D.N. Spergel,
\emph{Can small-scale baryon inhomogeneities resolve the Hubble tension?
      An investigation with ACT DR4}, 2021,\\
Phys Rev D 104, 063535,
\href{https://arxiv.org/abs/2105.03003}{\mbox{\texttt{arXiv:2105.03003 [astro-ph.CO]}}}

{\bf L. Thiele}, J.C. Hill, K.M. Smith,
\emph{Accurate Analytic Model for the Weak Lensing Convergence One-Point Probability
      Distribution Function and its Auto-Covariance}, 2020,\\
Phys Rev D 102, 123545,
\href{https://arxiv.org/abs/2009.06547}{\mbox{\texttt{arXiv:2009.06547 [astro-ph.CO]}}}

{\bf L. Thiele}, F. Villaescusa-Navarro, D.N. Spergel, D. Nelson, A. Pillepich,
\emph{Teaching neural networks to generate Fast Sunyaev Zel'dovich Maps}, 2020,\\
ApJ 902, 129,
\href{https://arxiv.org/abs/2007.07267}{\mbox{\texttt{arXiv:2007.07267 [astro-ph.CO]}}}

R. Cayuso, O.J.C. Dias, F. Gray, D. Kubiz\v{n}\'{a}k, A. Margalit, J.E. Santos,
R.G. Souza, {\bf L. Thiele},
\emph{Massive vector fields in Kerr--Newman and Kerr--Sen black hole spacetimes}, 2020,\\
JHEP 159,
\href{https://arxiv.org/abs/1912.08224}{\mbox{\texttt{arXiv:1912.08224 [hep-th]}}}

{\bf L. Thiele}, C.A.J. Duncan, D. Alonso,
\emph{Disentangling magnification in combined shear clustering analyses}, 2020,\\
MNRAS 491, 1746,
\href{https://arxiv.org/abs/1907.13205}{\mbox{\texttt{arXiv:1907.13205 [astro-ph.CO]}}}

R. Cayuso, F. Gray, D. Kubiz\v{n}\'{a}k, A. Margalit, R.G. Souza, {\bf L. Thiele},
\emph{Principal Tensor Strikes Again: Separability of Vector Equations with Torsion}, 2019,\\
Phys Lett B 795, 650,
\href{https://arxiv.org/abs/1906.10072}{\mbox{\texttt{arXiv:1906.10072 [hep-th]}}}

\newpage
\vspace*{0.5cm}

{\bf L. Thiele}, J.C. Hill, K.M. Smith, \emph{Accurate analytic model for the thermal
Sunyaev-Zel'dovich one-point probability distribution function}, 2019,\\
Phys Rev D 99, 103511,
\href{https://arxiv.org/abs/1812.05584}{\mbox{\texttt{arXiv:1812.05584 [astro-ph.CO]}}}

F. Dinc, M. Medvidovic, {\bf L. Thiele}, \emph{Effective Geometry Monte Carlo: A Fast and
Reliable Simulation Framework for Molecular Communication}, 2019,\\
IEEE Access 7, 28635

F. Dinc, {\bf L. Thiele}, B. C. Akdeniz, \emph{The effective geometry Monte Carlo
algorithm: applications to molecular communication}, 2019,\\
Phys Lett A 383, 2594,
\href{https://arxiv.org/abs/1809.06438}{\mbox{\texttt{arXiv:1809.06438 [cs.ET]}}}


\newpage
\vspace*{0.5cm}

\section{Academic \\ Honors}
Kusaka Memorial Prize in Physics (Princeton, 2022, \$3k)\\
Member of the German Academic Scholarship Foundation (2015 -- 2019, \$40k)\\
Perimeter Scholars International Award (Perimeter, 2018, \$34k)\\
Scott Prize for best performance in the 3rd year (Oxford, 2018, \$500)\\
%Offer for Part III of the Mathematical Tripos (Cambridge, 2018) - Declined\\
Winton Capital Prize for best performance in the 2nd year (Oxford, 2017, \$300)\\
%Gibbs Prize for practical work in the 2nd year (Oxford Physics, 2017)\\
BP Scholarship (Oxford, 2017, \$2.6k)\\
Rokos Award for summer research project (Oxford, 2016, \$1k)\\
%Distinction in Preliminary Examinations (Oxford University, 2016)\\
%Commendation for practical work (Oxford Physics, 2016)\\

\section{Professional \\ Service}
reviewer for ApJ, MNRAS, NeurIPS

\section{Talks}
\begin{tabular}{r l}
 5/20 & CCA Cosmo x ML \\
 5/20 & Princeton/IAS cosmo lunch \\
 5/20 & Perimeter Institute \\
 9/20 & German Astronomical Society \\
10/20 & MPA Garching \\
 8/21 & CMB-S4 meeting \\
 8/21 & Learn the Universe @ CCA \\
 1/22 & Cosmology Talks \\
 1/22 & AAS 239 \\
 3/22 & IAS astro coffee \\
 9/22 & UCL Physics \& Astronomy \\
 2/23 & Princeton gravity group \\
 3/23 & IPMU \\
 4/23 & Nagoya \\
 9/23 & Cosmo'23 Madrid \\
 9/23 & Institute d'Astrophysique Spatiale Orsay \\
 9/23 & IPMU CD3 seminar \\
10/23 & BCCP seminar UC Berkeley \\
10/23 & DESI lunch Berkeley Lab \\
10/23 & CMB constellation meeting KIPAC Stanford \\
10/23 & CCA Cosmo x ML tristate meeting \\
01/24 & AI4Phys @ IPMU \\
02/24 & Yale cosmology seminar \\
05/24 & MPA cosmology seminar \\
06/24 & LSS Quest Osaka \\
10/24 & Cosmo'24 Kyoto \\
10/24 & 11th KIAS workshop Gyeongju (invited) \\
12/24 & FAIRS-Japan workshop Nagoya (invited) \\
 2/25 & KICC, Cambridge UK \\
 2/25 & Voids @ CPPM, Marseille (invited) \\
 3/25 & FOPM seminar, UTokyo \\
 4/25 & LeCosPA Meets IPMU, Taipei \\
 5/25 & ML4Astro, UTokyo \\
 7/25 & AI for Fugaku seminar \\
 8/25 & 21st Rencontres du Vietnam (invited) \\
\end{tabular}


\section{Teaching}
\begin{itemize}
\item Astro-AI Asia Network (A3Net) summer school, 2024, Osaka, \emph{Basic Deep Learning} 
\item Lecture \& Tutorial for ILANCE students, 2025, IPMU, \emph{Introduction to Machine Learning}
\end{itemize}



\end{resume}
\end{document}
% BELOW NOT INCLUDED IN COMPILATION


\section{Research Experience}
	{\bf Master's Student,} Perimeter Institute \hfill 2018 -- present
	\begin{itemize} \itemsep -2pt  % reduce space between items
		\item With K.M. Smith \& J.C. Hill
		\item Computed thermal Sunyaev-Zel'dovich one-point function
		\item Included previously neglected effects (overlaps \& clustering)
		\item Developed Python code to evaluate theoretical model
		\item Direct significance for future datasets (e.g. advACT, Simons Observatory)
	\end{itemize}

	{\bf Project Student,} Oxford Department of Physics \hfill 2017 -- present
	\begin{itemize} \itemsep -2pt  % reduce space between items
		\item With D. Alonso \& C. Duncan
		\item Quantified the effectiveness of different data cutting schemes\newline in order to
		mitigate magnification bias in weak lensing galaxy surveys
		\item Used the \texttt{CLASS} package and Fisher matrix technique
	\end{itemize}
	
	{\bf Summer Student,} Dr Karl Remeis Observatory, Bamberg, Germany \hfill 2016
	 \begin{itemize} \itemsep -2pt  % reduce space between items
	 	\item With S. Falkner \& J. Wilms
		\item Fitted neutron star X-ray emission models to pulse profiles \newline from RXTE satellite data
		\item Improved epochfolding algorithm to find pulse periods
		%\item Presented work in group meetings
		\item Integrated code in X-ray package (\texttt{isisscripts})
	\end{itemize}
 
	{\bf Intern,} Astrophysical Institute \& University Observatory, Jena, Germany \hfill
	\mbox{2012 -- 2013}
	\begin{itemize} \itemsep -2pt %reduce space between items
		\item With M. Mugrauer
		\item Analyzed VLT/NACO images using ESO-MIDAS and ds9
		\item Searched for substellar and white dwarf companions via direct imaging
		\item Prepared paper for Astronomische Nachrichten
	\end{itemize}

\section{Teaching \\ Experience}
{\bf Seminar,} Landesschule Pforta, Germany \hfill 2018
\begin{itemize} \itemsep -2pt
	\item Gave a weekend seminar on theoretical mechanics and symmetries
	\item Encouraged gifted high school students to study physics
\end{itemize}

{\bf Teaching Lab,} Oxford Department of Physics \hfill 2018
\begin{itemize} \itemsep -2pt
	\item Built teaching experiment on quantized conductivity
	\item Wrote lab script and developed Python code for data analysis
\end{itemize}

\section{Work \\ Experience}
{\bf Content Creator,} EnergieWerk Ost GmbH, Dresden, Germany \hfill 2014 -- 2015
\begin{itemize} \itemsep -2pt %reduce space between items
	\item Increased reach of an online shop in a team of\\ programmers and writers
	\item Worked on Search Engine Optimization
\end{itemize}

\section{Further \\ Training}
	{\bf Spring Academy in Particle Physics and Cosmology,} Annecy \& CERN \hfill 2018
	\begin{itemize} \itemsep -2pt  % reduce space between items
		\item Studied theories about baryogenesis
		\item Gave talk on gravitational wave emission from\\ cosmological phase transitions
	\end{itemize}
	
	{\bf Summer Academy in Cosmology and Particle Physics,} Krakow, Poland \hfill 2017
	\begin{itemize} \itemsep -2pt  % reduce space between items
		\item Studied GR, QFT, FLRW cosmology, inflation
		\item Presented seminal papers to the group
	\end{itemize}
	
	{\bf Summer School in Theoretical Physics,} Perm State University, Russia \hfill 2017
	\begin{itemize} \itemsep -2pt  % reduce space between items
		\item Worked 7 weeks in tutorials and classes
		\item Studied Fluid Dynamics, Ferromagnetism, Spin Waves,\\ Convection, MHD, classical Perturbation Theory
	\end{itemize}



\section{Undergraduate\\ Talks}
Gravitational Wave Emission from Cosmological Phase Transitions (Annecy, 2018)\\
A short derivation of Hawking radiation (Inscite Conference Oxford, 2017)\\
Timing analysis for pulsars observed with the RXTE satellite (Pembroke College, 2016)

\section{Extracurriculars}
	{\bf President,} Oxford University Space \& Astronomy Society \hfill 2017 -- 2018 \\
	{\bf Secretary \& Observation Officer} \hfill 2016 -- 2017
	\begin{itemize} \itemsep -2pt  % reduce space between items
		\item organization of weekly talks and observations
		\item administrative tasks
	\end{itemize}

	{\bf Leader,} Astronomy Club Landesschule Pforta \hfill 2012 -- 2014
	\begin{itemize} \itemsep -2pt  % reduce space between items
		\item organization of observations and study days
		\item responsibility for the observatory and telescope
	\end{itemize}

	{\bf Others:} drama, floorball, chess

\section{Skills}
\emph{programming languages:} Python, C, S-Lang\\
\emph{software:} Mathematica, ESO-MIDAS, ISIS, Linux, Slurm, \LaTeX\\
\emph{languages:} German (native), English (fluent), Latin (reading)

\section{GRE scores}
Physics 990 (94\%), V 163 (93\%), Q 167 (91\%), A 5.0 (92\%)

\end{resume} 
\end{document} 



